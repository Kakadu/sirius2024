\section{Язык выражений}

Абстрактный синтаксис:

\[
\begin{array}{rcll}
  \mathscr{E} & = & 0,\, 1,\, 2,\, \dots & \mbox{натуральные числа} \\
              &   & x,\, y,\, z,\, \dots & \mbox{переменные} \\
              &   & e_1 \oplus e_2 & \mbox{выражения с бинарными операциями}
\end{array}
\]

Бинарные операции:

\[
\begin{array}{rcl}
  \oplus & = & + \\
  &   & - \\
  &   & * \\
  &   & / \\
  &   & \% \\
  &   & > \\
  &   & \ge \\
  &   & < \\
  &   & \le \\
  &   & = \\
  &   & \ne \\
  &   & \& \\
  &   & | 
\end{array}
\]

Абстрактный синтаксис определяет множество упорядоченных помеченных \underline{абстрактных синтаксических деревьев}.

Конкретный синтаксис: правила представления абстрактных синтаксических деревьев в каком-либо носителе (например, в
виде последовательного текста).

Семантика:

\setarrow{\xRightarrow}

\[
\trans{\sigma}{n}{n}\ruleno{Const}
\]
\[
\trans{\sigma}{x}{\sigma\,x}\ruleno{Var}
\]
\[
\trule{\trans{\sigma}{e_1}{z_1}\quad\trans{\sigma}{e_2}{z_2}}
      {\trans{\sigma}{e_1\oplus e_2}{z_1\otimes z_2}}\ruleno{BinOp}
\]

Соответствие между ``$\oplus$'' и ``$\otimes$'':

\[
\begin{array}{c|l}
  \oplus & \otimes \\
  \hline \\
  +   & \mbox{сложение} \\
  -   & \mbox{вычитание} \\ 
  *   & \mbox{умножение} \\
  /   & \mbox{частное} \\
  \%  & \mbox{остаток} \\
  >   & \mbox{1, если левый операнд больше правого, 0 иначе} \\
  \ge & \mbox{1, если левый операнд не меньше правого, 0 иначе} \\
  <   & \mbox{1, если левый операнд меньше правого, 0 иначе} \\
  \le & \mbox{1, если левый операнд не больше правого, 0 иначе} \\
  =   & \mbox{1, если левый операнд равен правому, 0 иначе} \\
  \ne & \mbox{1, если левый операнд не равен правому, 0 иначе} \\
  \&  & \mbox{минимум операндов; не определено, если хотя бы один из операндов отличен от 0 или 1} \\
  |   & \mbox{максимум операндов; не определено, если хотя бы один из операндов отличен от 0 или 1}  
\end{array}
\]

\section{Абстрактная стековая машина}

Инструкции:

\[
\begin{array}{rcl}
  \mathcal{I} & = & \cd{CONST}\;n\\
              &   & \cd{LD}\;x\\
              &   & \cd{BINOP}\;\oplus
\end{array}
\]

Программы:

\[
\begin{array}{rcl}
  \mathcal{P} & = & \epsilon \\
              &   & i\,p
\end{array}
\]

Семантика:

\[
\trans{\inbr{st,\,s}}{\epsilon}{\inbr{st,\,s}}\ruleno{Empty}
\]
\[
\trule{\trans{\inbr{st,\,(st\,x)s}}{p}{c}}
      {\trans{\inbr{st,\,s}}{[\cd{LD}\,x]p}{c}}\ruleno{LD}
\]
\[
\trule{\trans{\inbr{st,\,ns}}{p}{c}}
      {\trans{\inbr{st,\,s}}{[\cd{CONST}\,n]p}{c}}\ruleno{CONST}
\]
\[
\trule{\trans{\inbr{st,\,(y\otimes x)s}}{p}{c}}
      {\trans{\inbr{st,\,xys}}{[\cd{BINOP}\oplus]p}{c}}\ruleno{BINOP}
\]      

\section{Свойства семантики}

\begin{enumerate}
\item Детерминизм: если $\trans{st}{e}{n}$ и $\trans{st}{e}{m}$ то $n=m$.

\item Строгость: если для произвольного $e^\prime\preceq e$ и произвольного $st$
$\trans{st}{e^\prime}{\bot}$ то $\trans{st}{e}{\bot}$.

  Здесь ``$\preceq$''~--- отношение ``быть подвыражением``.
\end{enumerate}
